% Template:     Informe LaTeX
% Documento:    Archivo principal
% Versión:      8.1.7 (24/07/2022)
% Codificación: UTF-8
%
% Autor: Pablo Pizarro R.
%        pablo@ppizarror.com
%
% Manual template: [https://latex.ppizarror.com/informe]
% Licencia MIT:    [https://opensource.org/licenses/MIT]

% CREACIÓN DEL DOCUMENTO
\documentclass[
	spanish, % Idioma: spanish, english, etc.
	letterpaper, oneside
]{article}

% INFORMACIÓN DEL DOCUMENTO
\def\documenttitle {Informe 01: Caso de estudio}
\def\documentsubtitle {}
\def\documentsubject {\LARGE Amazon }

\def\documentauthor {Nombre del autor}
\def\coursename {Computación en la Nube y Big Data}
\def\coursecode {MCC-03}

\def\universityname {Universidad Nacional de San Agustín}
\def\universityfaculty {Escuela de Posgrado}
\def\universitydepartment {Maestría en Ciencia de la Computación}
\def\universitydepartmentimage {departamentos/logo_unsa}
\def\universitydepartmentimagecfg {height=1.57cm}
\def\universitylocation {Arequipa, Perú}

% INTEGRANTES, PROFESORES Y FECHAS
\def\authortable {
\begin{tabular}{ll}
	Integrantes:
	& \begin{tabular}[t]{l}
		Fredy Abel Huanca Torres \\
		Henrry Ivan Arias Mamani \\
		Jose Edison Pérez Mamani
	\end{tabular} \\
	Profesor:
	& \begin{tabular}[t]{l}
		Alvaro Henrry Mamani Aliaga
	\end{tabular} \\
	\multicolumn{2}{l}{Fecha de realización: \today} \\
	\multicolumn{2}{l}{Fecha de entrega: \today} \\
	\multicolumn{2}{l}{\universitylocation}
\end{tabular}
}

% IMPORTACIÓN DEL TEMPLATE
\input{template}

% INICIO DE PÁGINAS
\begin{document}
	
% PORTADA
\templatePortrait

% CONFIGURACIÓN DE PÁGINA Y ENCABEZADOS
\templatePagecfg

% RESUMEN O ABSTRACT
\begin{abstractd}
	El presente, es un informe sobre un caso de estudio: Amazon inc. el mismo que ha usado el framework Hadoop para el procesamiento y almacenamiento de los datos que generan, son soportados por artículos de investigación y datos obtenidos de recursos de internet, el cual ha servidor para realizar este informe, asimismo consideramos el volumen, variedad, velocidad y veracidad de los datos que procesa.
\end{abstractd}

% TABLA DE CONTENIDOS - ÍNDICE
\templateIndex

% CONFIGURACIONES FINALES
\templateFinalcfg

% ======================= INICIO DEL DOCUMENTO =======================

% Template:     Informe LaTeX
% Documento:    Archivo de ejemplo
% Versión:      8.1.7 (24/07/2022)
% Codificación: UTF-8
%
% Autor: Pablo Pizarro R.
%        pablo@ppizarror.com
%
% Manual template: [https://latex.ppizarror.com/informe]
% Licencia MIT:    [https://opensource.org/licenses/MIT]

\section{Introducción}

Amazon.com,  es una empresa estadounidense de comercio electrónico y computación en la nube, fue fundada en julio de 1994. Es conocida por la tienda más grande del mundo tiene; libros, piezas de repuesto para automóviles, juguetes para niños, productos electrónicos, etc. También es conocida por fabricar artículos de consumo. electrónica: Amazon Kindle, Amazon Alexa, Echo y muchos más. Amazon también permitió que los autores y editores publicaran y pusieran a disposición sus libros en Kindle Store, con el brazo de publicación "Amazon Publishing". \scite{jopson2011amazon}

% ------------------------------------------------------------------------------
% REFERENCIAS, revisar configuración \stylecitereferences
% ------------------------------------------------------------------------------
\clearpage
\bibliography{library}  % Ejemplo, se puede borrar

% FIN DEL DOCUMENTO
\end{document} 